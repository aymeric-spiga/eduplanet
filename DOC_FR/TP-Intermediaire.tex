\documentclass[a4paper,12pt]{article}
\usepackage[utf8]{inputenc}
\usepackage[frenchb]{babel}
\usepackage{setspace}
\usepackage[pdftex]{graphicx}
\usepackage{amsmath}
\usepackage{amssymb}
\usepackage{xcolor}
\usepackage{natbib}
\usepackage{hevea}
\usepackage{geometry}
\usepackage{enumitem}
\setlist[enumerate]{leftmargin=*,labelindent=3mm}
\setlist[itemize]{leftmargin=*,labelindent=3mm}
\graphicspath{{img/}}
\geometry{top=1cm,bottom=2cm,left=1cm,right=1cm}
\begin{document}
%TEXFILE_JBM_BEGIN

\title{\textbf{TP intermédiaire}\\
Utilisation du modèle \texttt{eduplanet}}
\author{\normalsize{Jean-Baptiste Madeleine et Aymeric Spiga}}
\date{Lundi 6 mars 2017}

\maketitle

\renewcommand{\labelitemi}{\textbullet}

\section{Introduction}

Nous allons dans ce TP mener des \textbf{études de sensibilité} du modèle à
deux paramètres: l'inertie thermique de la surface puis l'épaisseur optique en
gaz $\tau_s$. De façon générale, une étude de sensibilité consiste à quantifier
la réponse d'un modèle à un changement de ses paramètres d'entrée. La méthode
la plus simple est de réaliser d'abord une \textbf{expérience de référence},
puis de faire une nouvelle expérience en changeant un seul paramètre pour
constater les changements résultants. Pour mener une telle étude avec le modèle
\texttt{eduplanet}, la démarche sera toujours la suivante:

% REMARQUE: pour couper les texttt, utiliser des soft hyphen "\-"
\begin{itemize}
\item Lancer une première expérience de référence;
\item Ouvrir ensuite avec un éditeur de texte un fichier de réglage (fichiers
\texttt{reglages\_init.txt} ou \texttt{regla\-ges\_run.txt} du dossier \texttt
{eduplanet}) puis remplacer l'ancienne valeur du paramètre par la nouvelle
valeur en veillant à 1) bien préserver la structure du fichier, et 2) bien
modifier les fichiers qui se trouvent dans \texttt{eduplanet} et non dans les
dossiers d'expérience \texttt{expnum};
\item Enregistrer les fichiers et les nouveaux réglages;
\item Relancer l'expérience grâce à la commande \texttt{./run.sh};
\item Comparer les résultats des deux expériences (voir par exemple le script
\texttt{./TOOLS/compare.sh}).
\end{itemize}

Pour chaque expérience de sensibilité, on veillera à noter les paramètres
modifiés et les résultats obtenus. Notons également que les fichiers de
réglages \texttt{reglages\_init.txt} et \texttt{reglages\_run.txt} sont
sauvegardés lors de chaque expérience dans le dossier de l'expérience
\texttt{expnum\_DATE}. Cette sauvegarde permet de consulter les paramètres
d'une expérience passée si besoin\footnote{\textbf{Attention:} pour des raisons
techniques, il faut toujours régler les fichiers de réglages \texttt
{reglages\_init.txt} et \texttt{reglages\_run.txt} se trouvant dans le dossier
\texttt{eduplanet} à la main. Veiller à ne pas copier les fichiers de réglages
sauvegardés dans un dossier
\texttt{expnum\_DATE} dans le dossier \texttt{eduplanet} en écrasant les
fichiers de réglage originaux;}.


\section{Sensibilité du modèle à l'inertie thermique}

Considérons comme expérience de référence l'expérience réalisée au début du TP
d'introduction (veiller à reprendre le cas initial terrestre).
L'inertie thermique de la surface utilisée était de 12000~J~K$^{-1}$~m$^{-
2}$~s$^{-1/2}$ (unité qui sera notée par la suite tiu). Ouvrir un terminal et
se replacer dans le dossier \texttt{eduplanet} par la commande \texttt{cd
WORK/eduplanet}.

\begin{enumerate}

\item Afin de mieux observer l'évolution temporelle des résultats du modèle,
augmenter la résolution temporelle du fichier de résultat \texttt{resultat.nc}.
Pour cela, se placer dans le dossier \texttt{eduplanet}, ouvrir le fichier
\texttt{reglages\_run.txt}, et vérifier que le paramètre \texttt{ecritphy} est
à 18\footnote{Cela signifie que le modèle écrira les résultats tous les 18 pas
de temps;}. Une fois le paramètre \texttt{ecritphy} modifié, relancer
l'expérience de référence pour $I=12000$~tiu en tapant \texttt{./run.sh}.
Afficher les résultats à l'aide de la commande \texttt{./TOOLS/atlas.py},
constater l'amélioration de la résolution temporelle des résultats et
sauvegarder chaque figure dans le dossier d'expérience.

\item \`A présent, modifier l'inertie thermique de la surface pour créer une
planète désertique couverte de sable ($I=50$~tiu). Relancer l'expérience
numérique (en gardant le paramètre \texttt{ecritphy} à 18) et afficher ses
résultats à l'aide de la commande \texttt{./TOOLS/atlas.py}. Comparer la
Figure~1 obtenue pour $I=50$~tiu à celle obtenue pour $I=12000$~tiu. Quel est
l'impact du changement d'inertie thermique sur la température de surface et
l'amplitude de ses variations ?

\item Observer sur la Figure~1 l'évolution de la température de surface à
80$^\circ$S. Quand est-ce-que la température est maximale? Pourquoi la
température est-elle si élevée malgré la haute latitude?

\item Analyser à présent la Figure~2. Quelle est la durée de la réponse
transitoire du système (approximativement) ? Comment varie-t-elle avec
l'inertie thermique?


\item Analyser à présent le cycle saisonnier de la Figure~3. Les saisons dans
chaque hémisphère sont-elles plus ou moins marquées? La température de surface
$T_s$ (en bas à droite) s'approche-t-elle ou s'éloigne-t-elle des variations du
flux solaire absorbé par la surface $F_{vis}^{\downarrow_{Surf}}$ (en haut à
droite) lorsque I diminue?

\item Sur la Figure~4, à quelle latitude l'atmosphère est-elle la plus chaude
pour les deux saisons? Comment ceci varie-t-il avec l'inertie thermique?

\end{enumerate}

\section{Sensibilité du modèle à l'épaisseur optique de gaz $\tau_s$}
%-------------------------------------------------------------------

L'épaisseur optique d'un gaz dans la totalité de l'atmosphère\footnote{$\tau_s$
est aussi appelée opacité totale ou ``column opacity'' en anglais;} $\tau_s$
peut s'écrire:

\begin{equation}
\tau_s= k \ \frac{ r_X M_X }{M} \frac{P_s}{g},
\label{eqn-tausurf}
\end{equation}

où $k$ est le coefficient d'absorption infrarouge (aussi appelé section
efficace d'absorption massique), $r_X$ le rapport de mélange volumique (i.e.
molaire, en ppmv par exemple) du constituant X, $M_X$ la masse molaire du
constituant X, $M$ la masse molaire de l'atmosphère et $P_s$ la pression
atmosphérique en surface. Cette opacité totale en gaz peut être modifiée dans
le modèle par l'intermédiaire d'un paramètre nommé \texttt{kappa\_IR} se
trouvant dans le fichier \texttt{reglages\_run.txt}.

\begin{enumerate}

\item Sachant que \texttt{kappa\_IR}~$=k \ r_X$, déduire de l'équation~\ref
{eqn-tausurf} l'expression littérale du paramètre \texttt{kappa\_IR}. Le
calculer pour X~$\equiv$~CO$_2$ et $\tau_s=0.8$ (le reste des paramètres peut
être issu des fichiers de réglage du modèle). Constater que cette valeur
correspond bien à celle déjà renseignée dans le fichier \texttt
{reglages\_run.txt}. Calculer la valeur du coefficient d'absorption
$k$~=~\texttt{kappa\_IR}$/r_X$ correspondant à $r_X=300$~ppmv.

\item On élève à présent le rapport de mélange volumique de CO$_2$ de 300~ppmv
(valeur pré-industrielle) à 400~ppmv (valeur actuelle). Calculer la nouvelle
valeur du paramètre \texttt{kappa\_IR}~$=k \ r_X$ en préservant le coefficient
d'absorption $k$ trouvée à la question précédente. Réaliser une expérience avec
cette nouvelle valeur de \texttt{kappa\_IR} en modifiant le fichier \texttt
{regla\-ges\_run.txt}. De combien s'élève la température de surface moyenne à
l'équilibre du modèle ?

\item Nous comparons à présent les résultats du modèle avec nos calculs
analytiques. Au TP d'introduction, nous avions calculé $T_s$ pour $\tau_s=0.8$.
Calculer la nouvelle valeur de $\tau_s$ lorsque $r_X = 400$~ppmv et non
300~ppmv. Recalculer la valeur de $T_s$ correspondante.

\item L'élévation de température prévue par le modèle est-elle en accord avec
celle calculée ci-dessus ? Est-elle réaliste ? Quelle est la principale
hypothèse faite dans notre expérience pouvant expliquer ce biais par rapport à
la réalité ?

\end{enumerate}


%TEXFILE_JBM_END
\end{document}
