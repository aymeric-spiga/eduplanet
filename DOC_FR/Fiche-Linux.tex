\documentclass[a4paper,10pt]{article}
\usepackage[utf8]{inputenc}
\usepackage[frenchb]{babel}
\usepackage{setspace}
\usepackage[pdftex]{graphicx}
\usepackage{amsmath}
\usepackage{amssymb}
\usepackage{xcolor}
\usepackage{natbib}
\usepackage{hevea}
\usepackage{footnote}
\usepackage{geometry}
\usepackage{enumitem}
\setlist[enumerate]{leftmargin=*,labelindent=3mm}
\setlist[itemize]{leftmargin=*,labelindent=3mm}
\graphicspath{{img/}}
\geometry{top=0.cm,bottom=1cm,left=1cm,right=1cm}
\begin{document}
%TEXFILE_JBM_BEGIN

\title{\textbf{Fiche pratique n$^\circ$1}\\
Environnement Linux}
\author{\normalsize{Jean-Baptiste Madeleine}}
\date{}

\maketitle

\thispagestyle{empty}

Ce document résume les principales commandes du ``shell'', qui est l'interface
entre vous et le système d'exploitation Linux\footnote{Linux est un système de
type Unix, tout comme iOS ou bien Android, utilisés sur la plupart des
smartphones;} (d'où le nom de ``coquille'' du système). Le shell est accessible
par un terminal (aussi appelé console). Les logiciels \texttt{eduplanet} et
\texttt{planetoplot} ont été développés pour
être utilisés à partir du shell. Il est donc confortable de manipuler également
les fichiers liés aux expériences à partir du shell. Cependant, l'usage de la
souris et de l'interface graphique de Mandriva est aussi possible, selon votre
préférence.

\paragraph{Pour lancer un terminal,} une fois le bureau Mandriva\footnote{La
distribution de Linux que nous utilisons;} ouvert, cliquer sur le menu Démarrer
(étoile en bas à gauche de l'écran) puis sur ``Outils $>$ Konsole''. Vous
pouvez créer un raccourci en cliquant-déplaçant l'icône sur votre bureau.

\begin{savenotes}
\begin{table}[h!]
\centering
\begin{tabular}{lll}
Commande & Fonction & Exemple \\ \hline
%-------------------------------------------------------------------
\texttt{gedit \&} ou \texttt{kwrite \&} & Lance un éditeur de texte
& \texttt{gedit reglages\_init.txt \&} ouvre le fichier de
réglage\footnote{Le \texttt{\&} à la fin permet de récupérer la main sur le
terminal tout en lançant l'éditeur en fond;} \\
%-------------------------------------------------------------------
\texttt{xview \&} & Ouvre une image &
\texttt{xview expnum\-\_DATE\--ET\--HEURE/fig-Ts-Latitude-Ls.png \&} \\
& & affiche la figure \texttt{fig-Ts-Latitude-Ls.png} \\
%-------------------------------------------------------------------
\texttt{pwd} & Print Working Directory & \texttt{pwd} dans le dossier
WORK affichera \\
& & le chemin complet vers WORK \\
%-------------------------------------------------------------------
\texttt{ls} & Liste les fichiers & \texttt{ls} seul affiche les
fichiers du dossier courant \\
& & \texttt{ls WORK} affiche les fichiers du dossier \texttt{WORK} \\
%-------------------------------------------------------------------
\texttt{history} ou touche $\uparrow$ & Historique des commandes &
Permet de rappeler une commande déjà tapée\footnote{\'Egalement,
le copier-coller se fait en sélectionnant avec la souris la chaîne de
caractères à copier, puis en collant avec la molette;} \\
%-------------------------------------------------------------------
\texttt{more} & Aperçu d'un fichier & \texttt{more reglages\_init.txt}
affiche le contenu du fichier \\
& & \texttt{reglages\_init.txt} (barre espace pour défiler)\footnote{La
commande \texttt{more} est pratique pour afficher le contenu d'un fichier sans
risquer d'écrire dedans;} \\
%-------------------------------------------------------------------
\texttt{cd} & Change Directory & \texttt{cd eduplanet} vous emmène dans
le dossier \texttt{eduplanet} \\
%-------------------------------------------------------------------
\texttt{cp} & Copy & \texttt{cp reglages\_init.txt
reglages\_init.txt.sav} permet \\
& & de copier le fichier de réglages dans un nouveau fichier \\
& & \texttt{reglages\_init.txt.sav} pour le sauvegarder \\
%-------------------------------------------------------------------
\texttt{cp -r} & Copier Répertoire &
\texttt{cp -r eduplanet eduplanet-sav} copie tout le \\
& & dossier \texttt{eduplanet} dans un autre dossier \\
%-------------------------------------------------------------------
\texttt{mv} & Move & Comme \texttt{cp} mais déplace au lieu de copier
\\
%-------------------------------------------------------------------
\texttt{rm} & Remove & \texttt{rm notes.txt} supprime le fichier
\texttt{notes.txt} \\
%-------------------------------------------------------------------
\texttt{mkdir} & Make Directory & \texttt{mkdir WORK} crée un
répertoire \texttt{WORK} dans le dossier courant \\
%-------------------------------------------------------------------
\texttt{clear} & ``Coup d'éponge'' & \texttt{clear} efface tout ce qui
est affiché sur le terminal \\
%-------------------------------------------------------------------
\texttt{man} & Aide d'une commande & \texttt{man ls} affiche toutes les
informations sur \\
& & la commande \texttt{ls} (touche \texttt{q} pour sortir) \\
%-------------------------------------------------------------------
touche \texttt{TAB} & Auto-complétion & \texttt{cd WO} puis \texttt
{TAB} complète \\
& & automatiquement par \texttt{cd WORK}\footnote{On peut appuyer
plusieurs fois sur \texttt{TAB} pour afficher tous les dossiers commençant par
les mêmes lettres;} \\ \hline
%-------------------------------------------------------------------
\end{tabular}
\end{table}
\end{savenotes}

\paragraph{Attention:} \'Eviter absolument d'utiliser les caractères génériques
de Linux pour vos noms de fichiers (espaces, signe \& et autres). Le plus
simple est d'utiliser les lettres majuscules, les minuscules sans accent et les
chiffres, ainsi que le symbole ``\texttt{\_}''.

\paragraph{L'exemple de série de commandes} suivant sera petit à petit naturel
pour vous:
\begin{savenotes}
\begin{table}[h!]
\centering
\begin{tabular}{rl}
%-------------------------------------------------------------------
\texttt{cd $\sim$/WORK/eduplanet} & va dans le dossier WORK/eduplanet qui
est dans votre \\
& dossier utilisateur (donné par $\sim$) \\
%-------------------------------------------------------------------
\texttt{cd ..} & remonte dans WORK (chemin relatif) \\
\texttt{cd eduplanet} & redescend dans \texttt{eduplanet} (chemin relatif)
\\
\texttt{cd $\sim$/WORK} & retourne dans \texttt{WORK} mais cette fois par
le chemin absolu \texttt{$\sim$/WORK} \\
\texttt{cd} & tout seul retourne dans votre dossier utilisateur $\sim$ (le
``home''\footnote{Ce ``home'' est un peu l'équivalent du dossier ``Mes
Documents'' de Windows;}) \\
\texttt{mkdir WORK} & crée un dossier \texttt{WORK} dans le dossier
courant \\
\texttt{mkdir WORK/eduplanet} & crée un sous-dossier dans WORK seulement
si WORK existe \\
\end{tabular}
\end{table}
\end{savenotes}

%TEXFILE_JBM_END
\end{document}
