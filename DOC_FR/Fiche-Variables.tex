\documentclass[a4paper,11pt]{article}
\usepackage[utf8]{inputenc}
\usepackage[frenchb]{babel}
\usepackage{setspace}
\usepackage[pdftex]{graphicx}
\usepackage{amsmath}
\usepackage{amssymb}
\usepackage{xcolor}
\usepackage{natbib}
\usepackage{hevea}
\usepackage{footnote}
\usepackage{geometry}
\usepackage{enumitem}
\setlist[enumerate]{leftmargin=*,labelindent=3mm}
\setlist[itemize]{leftmargin=*,labelindent=3mm}
\graphicspath{{img/}}
\geometry{top=1cm,bottom=1cm,left=1cm,right=1cm}
\begin{document}
%TEXFILE_JBM_BEGIN

\title{\textbf{Fiche pratique n$^\circ$2}\\
Analyse des résultats du modèle \texttt{eduplanet}}
\date{}

\maketitle

\thispagestyle{empty}

Ce document résume les principales variables (pronostiques ou diagnostiques)
disponibles dans le fichier de résultat de l'expérience numérique \texttt
{resultat.nc}.

\section{Variables dynamiques}

\begin{savenotes}
\begin{table}[h!]
\centering
\begin{tabular}{lllll}
Variable & Notation usuelle & Nom dans \texttt{resultat.nc} & Unité
\\ \hline

Température de surface & $T_s(x,y,t)$ & \texttt{tsurf} & K \\
Pression de surface & $p_s(x,y,t)$ & \texttt{ps} & Pa \\
Pression atmosphérique & $p(x,y,z,t)$ & \texttt{p} & Pa \\
Température de l'atmosphère & $T(x,y,z,t)$ & \texttt{temp} & K \\
Température potentielle & $\theta(x,y,z,t)$ & \texttt{teta} & K \\
Vent zonal & $u(x,y,z,t)$\footnote{Le vent $\vec{v}$ s'écrivant suivant les
trois directions de l'espace $\vec{v}=u\ \hat{x}+v\ \hat{y}+w\ \hat{z}$;} &
\texttt{u} & m~s$^{-1}$ \\
Vent méridien & $v(x,y,z,t)$ & \texttt{v} & m~s$^{-1}$ \\
Vent vertical & $w(x,y,z,t)$ & \texttt{w} & m~s$^{-1}$ \\


\end{tabular}
\end{table}
\end{savenotes}

\section{Variables du transfert radiatif}

\begin{savenotes}
\begin{table}[h!]
\centering
\begin{tabular}{lllll}
Variable & Notation usuelle & Nom dans \texttt{resultat.nc} & Unité
\\ \hline
Albédo & $A_b(x,y,t)$ & \texttt{ALB} & \\
Flux solaire incident au sommet & $F_{vis}^{\downarrow_{TOA}}(x,y,t)$ &
\texttt{ISR} & W~m$^{-2}$ \\
Flux solaire absorbé par la surface & $F_{vis}^{\downarrow_{Surf}}(x,y,t)$
& \texttt{ASR} & W~m$^{-2}$ \\
Flux infrarouge émis au sommet & $F_{IR}^{\uparrow_{TOA}}(x,y,t)$ &
\texttt{OLR} & W~m$^{-2}$ \\
Flux géothermique à la surface & $F_q^{\uparrow_{Surf}}(x,y,t)$ &
\texttt{GND} & W~m$^{-2}$ \vspace{2mm} \\
Taux de chauffage SW\footnote{SW signifie shortwave, donc correspond au domaine
des courtes longueurs d'onde (appelé aussi domaine ``visible'');} & $Q_
{rad}^{SW}(x,y,z,t)$ & \texttt{zdtsw} & K~s$^{-1}$ \vspace{2mm} \\
Taux de chauffage LW\footnote{De la même façon LW signifie longwave, et
correspond au domaine des grandes longueurs d'onde ou domaine infrarouge);}
& $Q_{rad}^{LW}(x,y,z,t)$ & \texttt{zdtlw} & K~s$^{-1}$ \vspace
{2mm} \\
Taux de chauffage total & $Q_{rad}(x,y,z,t)$ & \texttt{dtrad} &
K~s$^{-1}$ \vspace{2mm} \\
\end{tabular}
\end{table}
\end{savenotes}
\section{Variables des paramétrisations}


\begin{savenotes}
\begin{table}[h!]
\centering
\begin{tabular}{lllll}
Variable & Notation usuelle & Nom dans \texttt{resultat.nc} & Unité
\\ \hline
RMM\footnote{RMM signifie rapport de mélange massique, en kg de l'espèce
considérée par kg d'air;} en vapeur d'eau & $q_{vap}(x,y,z,t)$ &
\texttt{h2o\_vap} & kg~kg$^{-1}$ \\
RMM en glace d'eau & $q_{glace}(x,y,z,t)$ & \texttt{h2o\_ice} &
kg~kg$^{-1}$ \\
Masse de glace au sol par m$^2$ & $m_{glace}(x,y,t)$ & \texttt
{h2o\_ice\_surf} & kg~m$^{-2}$ \\
\end{tabular}
\end{table}
\end{savenotes}
%TEXFILE_JBM_END
\end{document}
